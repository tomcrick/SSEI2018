%%%% Proceedings format for most of ACM conferences (with the exceptions listed below) and all ICPS volumes.
\documentclass[sigconf]{acmart}
\usepackage{graphicx}
\usepackage{paralist}
\usepackage{booktabs}
\usepackage{tabularx}
\usepackage{url}
\usepackage[hyphenbreaks]{breakurl}

\def\UrlBreaks{\do\/\do-}

%%%% As of March 2017, [siggraph] is no longer used. Please use sigconf (above) for SIGGRAPH conferences.

%%%% Proceedings format for SIGPLAN conferences 
% \documentclass[sigplan, anonymous, review]{acmart}

%%%% Proceedings format for SIGCHI conferences
% \documentclass[sigchi, review]{acmart}

%%%% To use the SIGCHI extended abstract template, please visit
% https://www.overleaf.com/read/zzzfqvkmrfzn

\usepackage{booktabs} % For formal tables


% Copyright
%\setcopyright{none}
%\setcopyright{acmcopyright}
\setcopyright{acmlicensed}
%\setcopyright{rightsretained}
%\setcopyright{usgov}
%\setcopyright{usgovmixed}
%\setcopyright{cagov}
%\setcopyright{cagovmixed}

\copyrightyear{2018}
\acmYear{2018}
\setcopyright{acmlicensed}
\acmConference[SSEI2018]{Social Sensing and Enterprise Intelligence}{Apr. 23, 2018}{Lyon, France}
% ACM had "Computing" instead of Computer, please update
%\acmBooktitle{}
% \acmPrice{15.00}
% \acmDOI{10.1145/}
% \acmISBN{}
% This slight change to the code may also save 1 or 2 lines of space.

% removes the headers from each page per the preparation instructions, as these are not needed and will be updated with the chairs' actual session names during the pagination/indexing process:
\fancyhead{}

\begin{document}
\title[An Evaluation of Performance and Competition in Customer Services on
  Twitter]{An Evaluation of Performance and Competition in Customer Services on
  Twitter: A UK Telecoms Case Study}
%\titlenote{}
%\subtitle{Extended Abstract}
%\subtitlenote{}

\author{Nabeel Albishry}
%\orcid{}
\affiliation{%
  \institution{University of Bristol}
  \streetaddress{}
  \city{} 
  \country{United Kingdom}
}
\email{n.albishry@bristol.ac.uk}

\author{Tom Crick}
\orcid{0000-0001-5196-9389}
\affiliation{%
  \institution{Swansea University}
  \streetaddress{}
  \city{} 
  \country{United Kingdom}
}
\email{thomas.crick@swansea.ac.uk}

\author{Theo Tryfonas}

\affiliation{%
  \institution{University of Bristol}
  \streetaddress{}
  \city{} 
  \country{United Kingdom}
}
\email{theo.tryfonas@bristol.ac.uk}

\author{Tesleem Fagade}

\affiliation{%
  \institution{University of Bristol}
  \streetaddress{}
  \city{} 
  \country{United Kingdom}
}
\email{tesleem.fagade@bristol.ac.uk}


 
% The default list of authors is too long for headers}
\renewcommand{\shortauthors}{Albishry, Crick, Tryfonas, and Fagade}


\begin{abstract}
With an increasing number of consumers using social media platforms to
share both their satisfaction and displeasure about the products and
services they use, organisations with a customer service focus are
recognising the importance of genuine -- and rapid -- engagement with
their customers. In turn, consumers judge organisations on the quality
of customer service and degree of responsiveness to queries. This
paper presents an extensible framework for evaluating direct
engagements of customer service teams with customers. Furthermore, it
measures indirect engagement with industry sector rivals, competition
and their patterns, and intensity. By applying graph analysis of these
Twitter interactions, our framework was used to generate analytical
measures and visual representations for a case study based on seven
major UK telecoms companies. With a dataset consisting of 15,000
tweets and 3,500 user profiles, the results provide sustained evidence
for indirect engagements between business rivals, with customer
queries as the trigger for those competition, with competition more
intense between companies in same industry sub-domain.
\end{abstract}

\keywords{Customer services, Twitter, reply chains, graph
  construction, social network analysis, social media}

\maketitle

\section{Introduction}\label{intro}

The online news and social networking service Twitter has become one
of the most popular social platforms for a variety of demographics
across the world. It provides a rich, constantly updating, corpus of
big social data to study a range of complex socio-cultural issues,
from life event detection~\cite{blamey-et-al-2013} and identifying
multilingual communities~\cite{albishry-et-al:iccci2017}, through to
providing deeper insight into personality and
behaviour~\cite{mostafa-et-al-ai2016}. Unsurprisingly, Twitter is
increasingly being used by organisations to communicate with their
customers, due to the fast and convenient medium of
engagement~\cite{ma-et-al:2015}, using a variety of sophisticated
human and automated
approaches~\cite{verhagen-et-al:2014,xu-et-al:2017}. In 2016, a survey
was carried out on 5,450 people who follow small or medium-sized
enterprises (SME) on Twitter~\cite{Twitter2016}; the key results show
that 83\% of people that received a reply felt better about the SME,
and 68.7\% have made at least one purchase from an SME because of
Twitter.

However, this medium could serve as an indicator to underlying issues
of performance, management and even strategic
matters~\cite{gregoire-et-al:2015}; in many instances, the majority of
complaints deal with product and service-related
issues~\cite{einwiller+steilen:2015}. Many studies have been conducted
to explore aspects of customer services experiences in various
business domains, such as travel and
telecoms~\cite{Shakeel2017,Zhang2016,Wattimena2017,misopoulos-et-al:2014,Khatoon2017}.
News agencies are not far from social media analysis, they use it to
uncover users interests so they can provide more focused
contents~\cite{Nigam2016}. While various domains have long applied
network analysis techniques -- especially for crime detection and
prevention~\cite{oatley+crick:2015,oatley-et-al:dasc2015} -- only
recently has work has been conducted to see how users relate to brands
via network structures~\cite{Cutler2017}, how information shared by
companies disseminate and their types~\cite{Piccialli2017}, and what
type of engagements from companies was found to be of effect on
customers perception of the brand~\cite{Ibrahim2017}. A very common
approach in conducting such studies was using sentiment analysis,
mainly to measure consumer's perception and
satisfaction~\cite{Zhang2016,Al-Hussaini2017}.

However, the novel framework presented here aimes to provide
quantitative insights that can produce holistic views of customer
service accounts and interactions. Rather than focusing on individual
posts and their sentiment, the framework helps in identifying
important post conversations that can then be interrogated further by
analysts or decision makers. With the high volume of activity on
Twitter, the framework helps to easily identify key issues further
analysis. Furthermore, by using streaming and RESTful data, this
approach can be applied to live data to catch problematic
conversations before they reach certain thresholds.

The remainder of this paper is organised as follows: in
Section~\ref{method}, we introduce the methodology;
Section~\ref{results} presents the results and key visual
representations; Section~\ref{discussion} provides the main
discussion; Section~\ref{conclusions} concludes the paper with a
discussion of potential extensions and wider application of this work.


\section{Methodology}\label{method}

Inspired by the approach taken by Cogan et al.~\cite{Cogan2012}, this
study consists of two main steps: the data collection phases and the
graph construction. The data collection runs iteratively to obtain
reply chains, process them and save them to a database. Once the data
collection phase is completed, a large graph that includes all reply
nodes and edges is constructed to conduct the initial analysis. The
NetworkX Python package~\cite{Hagberg2008} was used for the graph
construction, while Gephi~\cite{Bastian2009} provided a range of tools
for visualisation.

% \begin{figure}[htb]
% \centering
% \includegraphics[width=\columnwidth]{images/frameworkstructure.png}
% \caption{Overall framework structure}
% \label{fig:frameworkstructure}
% \end{figure}

\subsection{Case Study: UK Telecoms Industry}

The dataset contains tweets and related replies for seven well-known
UK telecoms companies: BT, EE, giffgaff, O2, Sky, Virgin Media and
Vodaphone.  The choices were intended to represent companies of
various sizes, history and range of services provided. While a few
companies were found with one account on Twitter, some of them have
multiple accounts alongside the primary Twitter account. In those
instances, the dedicated customer services accounts were indicated in
the biography of the company's other accounts. Therefore, as the focus
of the study is on customer services on Twitter, data were collected
from either company's primary account or its dedicated customer
services one (N.B. names throughout the paper refer to Twitter account
handles rather than official company brand names).

\subsection{Streaming}

To ensure we were able to collect as much data as possible, the data
collection comprised of three steps. First, a stream endpoint is
opened to catch activities of accounts under investigation, those
accounts will be referred to as `watched' or `CS' (customer service)
accounts. The Twitter Streaming
API\footnote{\url{https://developer.twitter.com/en/docs}} is designed
to return tweets created by the user, their retweets, replies directed
to their tweets, and retweets of their tweets. However, the stream
does not include tweets mentioning the user, and replies/retweets by
protected users.

\subsection{Reply Chains}

Returned statuses from the Streaming API may represent reply to
statuses that have not been collected previously. It was found that
most missing statuses were either posted before the data collection
started, mentions, or that the user account is protected. Naturally,
this issue could have a huge impact on the quality of the analysis. 
Therefore, once statuses are returned from the stream endpoint, 
their type is checked first (tweet, retweet..etc). If it is a reply, 
the ID of the status to which it was replying is extracted from 
{\emph{in\_reply\_to\_status\_id}}. Then, using the extracted ID, we 
check if the replied-to status has already been collected and present 
in the dataset or not. If not, the REST API is then used to collect 
them. This process runs recursively for newly collected replies until 
no further replies are available. Unavailable statuses are often 
results from either deletion or protected accounts.

An analysis of changes on the graph after the second phase of data
collection shows that there were increases in the number of nodes and
edges by 43\% and 62\%, respectively. This increase in connections has
resulted in merging 176 components into others, which improved
connectivity of the graph and, thus, accuracy of the dependent
analyses.

% \begin{figure}[htb]
% \centering
% \includegraphics[width=\columnwidth]{images/datacollectionphases.png}
% \caption{(a) First phase of data collection and (b) second phase of data
% collection}
% \label{fig:datacollectionphases}
% \end{figure}


\subsection{Graph Construction}

\subsubsection{Main Graph}

Graph construction is the foundation of the analysis presented in this
paper.  First, a base graph\footnote{The terms `main graph' and `base
graph' are used interchangeably.} is generated containing all replies
and the needed data.  Nodes represent post IDs, while edges indicate
replying direction. Other information of statuses are added as
attributes to nodes. The information used in this study are screen
name of the user, timestamp of the reply, text, and `watched' entity,
to identify CS accounts; an example graph is shown in
Figure~\ref{fig:replychaingraph}. Properties of this graph are
constrained by how replies relate to each other. In other words, no
edge is expected to have weight value other than 1. Also, no reply
status can have outdegree greater than 1, however some nodes may have
an outdegree of 0. Node with outdegree=0 can be root nodes, i.e. first
post of conversation, or they were directed to unavailable
statuses. On the other hand, indegree of nodes can be 0 or
more. Special case nodes are those with indegree and outdegree equal
to 0. Those are isolated/floating nodes and must be removed before we
perform the analysis -- these nodes do not benefit the analysis as
they are not part of any conversation. Furthermore, they will be seen
as connected component by themselves, which impacts upon the accuracy
of results.

\begin{figure}[htb]
\centering
\includegraphics[width=\columnwidth]{images/replychaingraph.png}
\caption{Example of a reply chain graph}
\label{fig:replychaingraph}
\end{figure}

\subsubsection{Users' Graph}

Because most of the analysis focus on relationships between reply
posts, they were applied on the base graph. Nevertheless, to allow
examination of the relationships between users, another graph is
generated from the base graph. This process is carried out by
iterating through edges linking reply posts, extracting users'
information, and constructing users graph accordingly. In the context
of this study, only two attributes are used: screen names and
`watched' values. While nodes represent screen names, `watched' valued
are added as attribute of nodes. For edges, their weights indicate
number of replies sent from origin node (sender) to target node
(receiver), therefore, the user graph is directed. Applying this
process on the example in Figure~\ref{fig:replychaingraph} results in
the users graph shown in Figure~\ref{fig:usersgraph}.

\begin{figure}[htb]
\centering
\includegraphics[width=\columnwidth]{images/usersgraph.png}
\caption{Example of users' graph extracted from base graph}
\label{fig:usersgraph}
\end{figure}

To examine relationships between users, five network graph properties
are measured. There were no special case nodes or edges in this graph,
as observed in the base graph. Graph properties that are used in
analysing this graph and their contextual interpretations are shown in
Table~\ref{tbl:uucentralitymeasuresinter}.

\begin{table}[!h]
\centering
\begin{tabularx}{\columnwidth}{lX}
\toprule
\textbf{Measure} & \textbf{Interpretation} \\ 
\midrule
{\emph{Indegree}} & Number of users that sent reply to the node \\
{\emph{Weighted Indegree}} & Total number of received replies \\
{\emph{Outdegree}} & Number of users that have received a reply from
                     the node \\ 
{\emph{Weighted Outdegree}} & Total number of sent replies \\
{\emph{Edge Weight}}& Number of replies between the connecting nodes\\
\bottomrule
\end{tabularx}
\caption{User-user centrality measures interpretation}
\label{tbl:uucentralitymeasuresinter}
\end{table}

\subsection{Connected Components}

The base graph consists of many number of subgraphs, each of which
represent related replies, i.e. one conversation entity. Investigating
connected components plays a major role in identifying conversations
for accounts. They are used to measure size of conversation, their
depths, and to identify shared conversations between watched
accounts. To extract conversations that user or users were engaged in,
the process iterates through nodes in each component and examine the
`name' attributes. As soon as the search is matched, no further nodes
are examined. Then, the component is either analysed on the fly, or
returned if more intensive analysis is required.

In the base graph, the number of connected components reflect
conversations, while in the users' graph, connected components show
the users' communities. Therefore, in the base graph many components
should be expected, depending on activity of the watched
accounts. However, in the users' graph, the number of components
should not exceed the number of watched accounts, although there might
be exactly one component due to common customers.


\section{Results}\label{results}

\subsection{Accounts Activity}

An overall evaluation was carried to measure accounts activity. As
shown in Figure~\ref{fig:totalactivity}, {\emph{@virginmedia}} was
found the most active CS account by far. As the focus of this section
is on customer service, it is necessary to investigate post types to
examine the purposes of these accounts. The result shows that replies
were at least 83.5\% of accounts activity. This confirms that all
chosen accounts are primarily used to interact with customers,
handling requests and queries. Therefore, the resulting analyses will
be based on replies only.

\begin{figure}[htb]
\centering
\includegraphics[width=\columnwidth]{images/totalactivity.png}
\caption{Total activity}
\label{fig:totalactivity}
\end{figure}

% \begin{figure}[htb]
% \centering
% \includegraphics[width=\columnwidth]{images/accountsactivity.png}
% \caption{Accounts activity}
% \label{fig:accountsactivity}
% \end{figure}

\subsection{Interaction and Users}

Although the indicated ``working hours'' of CS accounts is important
to evaluate activity, it is valuable to measure posts that are
directed to those accounts from other users and examine them in line
with CS accounts. Audience and their relations with the accounts can
be analysed directly from the main graph (i.e. post-post). However, as
user details are embedded inside post nodes, observing such
relationships will not simple task to accomplish. Therefore, a
user-user graph was built from the main graph. The resultant graph
contains 3,521 user nodes and 5,938 edges, as shown in
Figure~\ref{fig:userusergraph}. Although post-post relationship cannot
have weight higher than one, user-user edge weight indicates number of
posts in one direction, which explains the reduction in number of
nodes and edges.

\begin{figure}[htb]
\centering
\includegraphics[width=\columnwidth]{images/userusergraph.png}
\caption{User-user graph}
\label{fig:userusergraph}
\end{figure}

Based on Table~\ref{tbl:uucentralitymeasuresinter}, summary of the
graph is presented below in Table~\ref{tbl:uucentralitymeasures}. The
table says that {\emph{@virginmedia}} received that highest number of
replies from 866 users with an average of 3.05 per user. Also, the
same account scored highest in number of recipients. The difference
between indegrees and outdegrees shows that apart from {\emph{@o2}},
all accounts have outdegrees bigger that their
indegrees. Additionally, the total number of sent replies is found to
be more than the number of received replies; this may reflect that
those replies were directed to non-reply posts.

\begin{table}[!h]
\centering
\begin{tabularx}{\columnwidth}{l|rrc|rrc}
\toprule
\textbf{Measure} & \textbf{ind} & \textbf{w.ind} & \textbf{\%} & \textbf{out} & \textbf{w.oud} & \textbf{\%}\\ 
\midrule
{\emph{btcare}} & 330 & 995 & 3.02 & 485 & 1317 & 2.72\\
{\emph{ee}} & 247 & 432 & 1.75 & 470 & 778 & 1.66 \\
{\emph{giffgagg}} & 77 & 209 & 2.71 & 102 & 247 & 2.42 \\ 
{\emph{o2}} & 293 & 463 & 1.58 & 260 & 479 & 1.84 \\
{\emph{skyhelpteam}} & 147 & 254 & 1.73 & 305 & 504 & 1.65\\
{\emph{virginmedia}} & 866 & 2645 & 3.05 & 1215 & 3421 & 2.82\\
{\emph{vodaphoneukhelp}} & 166 & 403 & 2.43 & 302 & 660 & 2.19\\
\bottomrule
\end{tabularx}
\caption{Centrality measures of user-user graph}
\label{tbl:uucentralitymeasures}
\end{table}

\subsection{Delay}\label{results_delay}

The observation of active hours provides an overall view of account
activity. However, calculating delays is important to provide more
insights on performance of CS team. As reply nodes in the base graph
(post-post) include timestamp attribute, measuring delay was achieved
by calculating time differences between end nodes on each
edge. Table~\ref{tbl:delaystats} shows descriptive statistics for CS
account delays. Interestingly, {\emph{@skyhelpteam}} was found with an
average delay of 45.04 hours, although rest of the CS accounts' delay
ranged between 1.14 and 3.34 hours.

% \begin{figure}[htb]
% \centering
% \includegraphics[width=\columnwidth]{images/delaymeans.png}
% \caption{Delay means for CS accounts}
% \label{fig:delaymeans}
% \end{figure}

\begin{table}[!h]
\centering
\begin{tabularx}{\columnwidth}{lrrrr}
\toprule
\textbf{Account} & \textbf{mean} & \textbf{stdev} & \textbf{max} & \textbf{min(sec)} \\ 
\midrule
{\emph{btcare}} & 2.04 & 16.11 & 572.46 & 38\\
{\emph{ee}} & 1.46 & 3.39 & 19.28 & 27\\
{\emph{giffgagg}} & 1.22 & 10.25 & 159.97 & 73\\ 
{\emph{o2}} & 1.14 & 2.66 & 22.48 & 58\\
{\emph{skyhelpteam}} & 45.04 & 49.16 & 117.21 & 44\\
{\emph{virginmedia}} & 3.34 & 9.25 & 263.98 & 22\\
{\emph{vodaphoneukhelp}} & 1.92 & 5.01 & 76.51 & 50\\
\bottomrule
\end{tabularx}
\caption{Summary delays statistics}
\label{tbl:delaystats}
\end{table}

\subsection{Conversation Components}

As covered in the methodology, each connected component in the
post-post graph represent a conversation entity that includes related
replies. In this dataset, there were 3,289 conversation components
with various number of replies. Observations of their sizes shows that
the smallest component consists of one post, while the biggest
component contains 81 posts.  Furthermore, the number of one-post
components was 102, and they were all found belonging to CS
accounts. Examining those singular components revealed that they were
either original tweets that have no replies or replies with no
replied-to post available. As covered earlier, missing replies are
those that could not be captured due to a deletion or their posting
account being protected.  Because they do not have any length, and
hence do not represent an examinable conversation, single-node
components will be excluded from forthcoming analyses.

Additionally, most common size of connected components was found to be
two nodes.  They were 1,188 components and the direction of their
edges revealed that most of these communications were from CS accounts
and directed to customer's post. However, 25 of those conversations
were initiated by customers.  As they are in two-node components, this
shows that those posts have not been answered by CS. Although other
means of communications could have been used, such as direct messages,
there were no visible sign of such interaction.

% \begin{figure}[htb]
% \centering
% \includegraphics[width=\columnwidth]{images/ccsizes.png}
% \caption{Connected component sizes}
% \label{fig:ccsizes}
% \end{figure}

\subsection{Component Size and Longest Path}

It is important to note that the size of connected components does not
necessarily reflect length of conversations, although there is a
strong correlation between size of component and length of its longest
path (0.88). As can be seen in Figure~\ref{fig:ccsizepaths}, many
components measures are positioned in a near perfect diagonal line;
interestingly, the longest path in the biggest component (81
nodes/posts) was only 1.

\begin{figure}[htb]
\centering
\includegraphics[width=\columnwidth]{images/ccsizepaths.png}
\caption{Size of components and their longest paths}
\label{fig:ccsizepaths}
\end{figure}

To further explore properties of connected components, the largest 20
components, as shown in Figure~\ref{fig:20ccpostpostgraph}, were
examined.  The findings show that components with very high variations
in indegree amongst their nodes are mostly originate from CS. Examples
of this claim is illustrated by the three big components in the
figure. When observed, they were found to featuring advertising
tweets. Another observation on the biggest component is that origin
node was a post by {\emph{@o2}} and has an indegree of 80, while rest
of the nodes have indegree of zero, i.e. they were not answered. On
the other hand, the longest path component was ranked the third
biggest component. It was found with a single leaf, and all other
nodes along the path were found with indegree=1 and outdegree=1,
forming what we call a {\emph{simple chain}}, uniquely coloured in
Figure~\ref{fig:20ccpostpostgraph}. Additionally, 15 of those
components were found to have originated from customer accounts, and
they all take a semi-simple chain.

\begin{figure}[htb]
\centering
\includegraphics[width=\columnwidth]{images/20ccpostpostgraph.png}
\caption{Largest 20 connected components in post-post graph}
\label{fig:20ccpostpostgraph}
\end{figure}

Generally, simple chains can be identified where the number of edges
equals length of the longest path in component. Simple chains account
for 80\% of the connected components in graph, of which 47\% were
found with the length of 1. This is in agreement with the results of
connected component sizes presented earlier. Finally,
Table~\ref{tbl:delaystatscl} presents statistics on chains of
individual CS accounts.

% \begin{figure}[htb]
% \centering
% \includegraphics[width=\columnwidth]{images/simplechainlengths.png}
% \caption{Simple chain lengths}
% \label{fig:simplechainlengths}
% \end{figure}

\begin{table}[!h]
\centering
\begin{tabularx}{\columnwidth}{lrrrrr}
\toprule
\textbf{Name} & \textbf{count} & \textbf{max} & \textbf{min} & \textbf{mean} & \textbf{stdev}\\ 
\midrule
{\emph{btcare}} & 388 & 19 & 1 & 3.46 & 3.22\\
{\emph{ee}} & 324 & 9 & 1 & 1.98 & 1.43\\
{\emph{giffgagg}} & 147 & 12 & 1 & 2.39 & 1.98\\ 
{\emph{o2}} & 216 & 11 & 1 & 2.53 & 2.26\\
{\emph{skyhelpteam}} & 252 & 15 & 1 & 2.22 & 1.99\\
{\emph{virginmedia}} & 959 & 68 & 1 & 3.78 & 4.46\\
{\emph{vodaphoneukhelp}} & 248 & 39 & 1 & 2.58 & 3.29\\
\bottomrule
\end{tabularx}
\caption{Summary statistics on chain length for CS accounts}
\label{tbl:delaystatscl}
\end{table}

\subsection{Coexistence and Competition}

As covered eariler, connected components in the base graph represent 
conversations. Therefore, those components were utilised to uncover 
indirect engagement amongst CS accounts. As reply nodes in the graph include 
screen name of user, it is possible to identify those components with 
more than one CS account. For each connected component, names in reply nodes
are checked if they belong to CS account or public. Components with more than
one distinct CS name are then marked as coexistence component. The results show
that there were 39 coexistence components in total; 38 include two CS accounts, and one
includes three accounts. The graph presented in
Figure~\ref{fig:commoncc} shows those components, with each CS account
given a colour code for identification as the legend clarifies.

\begin{figure*}[htb]
\centering
\includegraphics[width=0.6\textwidth]{images/commoncc.png}
\caption{Common connected components}
\label{fig:commoncc}
\end{figure*}

To explore these relationships further, a specific users graph was
constructed based on those components. Construction of this graph
follows similar steps as used in customer-CS graph. However, as CS
accounts do not have direct engagement with each other, edges in this
graph are undirected and their weights indicate number of times they
appeared together in the same conversation component.  The resulted
CS-CS graph is shown in Figure~\ref{fig:coexistencegraph}, where node
size indicates degree of node to show how many other CS accounts the
node has coexisted with, while darkness of node reflects weighted
degree to show the total frequency of coexistence for the node.

The first observation on the graph is that {\emph{@giffgaff}} account
was not found in any common conversation. In contrast, {\emph{@o2}}
was the only account that have shared conversations with all other CS
accounts, while {\emph{@vodafoneukhelp}} was found with the least
common conversations. Nevertheless, weighted degree measure shows that
{\emph{@virginmedia}} was the highest in number of common
conversations, 21 components, although its degree tells that those
conversations were shared with only three other CS teams. The heaviest
edge existed between {\emph{@virginmedia}} and {\emph{@btcare}},
followed by the edge between {\emph{@virginmedia}} and
{\emph{@skyhelpteam}}. Also, edges of {\emph{@o2}} show that it was
mostly appeared with {\emph{@ee}}, and for {\emph{@vodafoneukhelp}} it
was {\emph{@ee}}.

The observation of edges and their weights can provide insight into
uncovering more specific service areas within the specific industry or
sector. This was found clear when the modularity of the graph was
examined~\cite{Blondel2008}. The result has unfolded into two
communities, as shown in Figure~\ref{fig:modularityclassgraph}. Also,
industry knowledge regarding these following CS teams: {\emph{@ee}},
{\emph{@o2}}, and {\emph{@vodafoneukhelp}} belong to a domain that is
mostly focused on mobile services, while {\emph{@skyhelpteam}},
{\emph{@virginmedia}}, and {\emph{@btcare}} are mostly known to be
focusing on landline and home internet services.

\begin{figure}[htb]
\centering
\includegraphics[width=\columnwidth]{images/coexistencegraph.png}
\caption{CS-CS coexistence graph}
\label{fig:coexistencegraph}
\end{figure}

\begin{figure}[htb]
\centering
\includegraphics[width=\columnwidth]{images/modularityclassgraph.png}
\caption{Modularity class graph}
\label{fig:modularityclassgraph}
\end{figure}

Furthermore, using a similar approach that was used in
Section~\ref{results_delay}, the delay was measured in those
components to evaluate if presence of competitor has influence on how
quick CS team response. Interestingly, an improvement of 26\%, 43\%
and 72\% were observed for {\emph{virginmedia}},{\emph{btcare}} and
{\emph{skyhelpteam}}, respectively. Rather surprisingly, for the other
three companies, the average delay showed a drop by at least 9\%.

% \begin{figure}[htb]
% \centering
% \includegraphics[width=\columnwidth]{images/diffdelaymeans.png}
% \caption{Modularity class graph}
% \label{fig:diffdelaymeans}
% \end{figure}

% To better understand the context of conversations in those common
% components, text analysis was carried out for each CS account. The
% attempt is a try to find what the customer have said that made the CS
% team participate in the conversation. Therefore, posts of CS accounts
% were eliminated, and the analysed texts include posts of customers
% only. Presentation of the result could be shown in table of statics,
% rather, wordcloud visualisation was used for each CS account. Initial
% results show that mentions were found very high in number. Therefore,
% another one was generated after removing mentions, results are
% presented in Figure ‎2.17.

% Interestingly, the first set of results (with mentions) provide
% explanation and insights on edge weights shown in
% Figure~\ref{fig:20ccpostpostgraph}. Taking one example, the edge
% {\emph{virginmedia}}--{\emph{btcare}} was found the heaviest. However,
% the first set of wordclouds for both show that {\emph{virginmedia}}
% was the mostly mentioned by customers. Same conclusion was found for
% the edge {\emph{skyhelpteam}}--{\emph{virginmedia}}.

% The other set, with mentions removed, may be used to identify some
% keywords for what could possibly be main issues, problem, or complains
% that might be of advantageous to other business rivals.


\section{Discussion}\label{discussion}

Initially, the performance of CS accounts and their popularity on
Twitter were measured by an analysis of activity and users. From this
perspective, {\emph{@virginmedia}} was found to have the highest
volume of posts, the least diverse in terms of type of posts (99.7\%
were replies) and with the highest number of customers served.

The average delays of accounts ranged between 1.14 and
3.34 hours, apart from {\emph{@skyhelpteam}} which was found with an average
delay of 45.04 hours. This may indicate a management issue for the
team, such as unclear social media strategy or staff resources. 

Most CS teams have clearly specified working hours on their account
page, apart from {\emph{@giffgaff}} and {\emph{@o2}}. Interestingly,
these two accounts were found to have the lowest delay. Nevertheless,
high availability, i.e. longer activity hours, was not found to
significantly improve speed of reply to customers. For example, while
{\emph{@giffgaff}} was observed active for longer hours, {\emph{@o2}}
was found to be faster to reply.

Although the data shows that no CS team has been in a direct
engagement with a competitor, analysis of common/shared connected
components has uncovered some form of competition amongst CS
accounts. Particularly, in the case of {\emph{@virginmedia}} and
{\emph{@btcare}}, the competition was clear and intense. In all cases,
customers were found to be initiators of competing conversations by
making use of the Twitter {\emph{@-mention}} feature to bring
different rivals into conversations. In contrast to phone, letter or
email, complaints that are made on social media are open for the
public to read and follow, and therefore can be potentially damaging
to the business reputation if not handled appropriately. Therefore, it
was not surprising to see improvement in the speed of response in a
few instances where competitors were included in the same
conversation. This shows that with the openness of social media
platforms, such as Twitter, customers have more chance to obtain
better deals or speedy resolution of their
problems~\cite{einwiller+steilen:2015}. In turn, this nature of
publicly posted complaints add more pressure on CS teams to improve
their social media engagement, especially when business rivals are
included by customers~\cite{gregoire-et-al:2015}.

\section{Conclusions}\label{conclusions}

The paper has introduced an extensible framework for evaluating
customer service performance and competition between industry rivals
on Twitter. We have presented methods on how network graphs properties
can be used to make sophisticated evaluations, with the framework
being tested on selected accounts in the UK telecoms sector.

Section~\ref{method} highlighted two important techniques that need to
be applied prior to starting the analysis phase. First, the recursive
reply chain data collection is a crucial stage in obtaining accurate
results. The importance of this stage stems from the fact that it
fills the gaps and improve connectivity of graphs. Second,
construction of the initial graph from replies and the removal of
floating isolated nodes. In constructing this graph, key information
needs to be identified and attached as attributes to nodes. The
information used in this study include post {\emph{id}},
{\emph{timestamp}, {\emph{screen name}, {\emph{text}} and
{\emph{watched}} value. However, the framework could easily be
extended to include other information such retweets.

The core of this work was to show the importance of connected
components in distinguishing users' conversations, as well as
analysing competitions and their key features.  With the added value
of modularity classes, competition analysis has helped in uncovering
more specialist communities within the industry sector.

The presented framework could also be used by service providers to
reflectively evaluate their social media accounts and interactions, as
well as to generate insight into the activities of their key domain
competitors; in this way, the presented methods in this study could be
used to make real-time observations. Another application would be to
identify gaps, competitions, challenges and opportunities in services
that can be used in developing strategies for start-ups, for
example. The approach could also be applied to other domains or
contexts, such non-profit or the public sector. Also, it can be used
for groups of users, such as celebrities and their direct and indirect
engagements on Twitter.  Moreover, with the emerging practice of
signing a reply with a team member's initials, this practice can be
exploited to further augment this framework's capabilities; this
extension could help in estimating team sizes, working shifts and to
evaluate performance of individual team members.

\section*{Acknowledgements}

This work has been supported by a doctoral research scholarship for
Nabeel Albishry from King Abdulaziz University, Kingdom of Saudi
Arabia.


\bibliographystyle{ACM-Reference-Format}
\bibliography{ssei2018} 

\end{document}
